\chapter{Directorio \'Unico}

\section{Surgimiento del Directorio \'Unico}

Como resultado de la necesidad de acceso a la informaci\'on de los usuarios de la red de la Universidad de la Habana, tanto estudiantes, como profesores y externos, varios servicios han sido implementados de manera aislada, durante los \'ultimos años con el fin de suplir esta necesidad. A ra\'iz del proceso de centralizaci\'on de los servicios llevado a cabo en el per\'iodo de <fechainicio-fechafinal>\todo[color=yellow]{En qu\'e fecha se hizo esto?}, dichos servicios, relacionados con la creaci\'on, almacenamiento y la actualizaci\'on de la informaci\'on referente a los usuarios, fueron integrados en lo que se conoce hoy como Directorio \'Unico.

Entre las principales funcionalidades que componen el Directorio \'Unico, se encuentra la de constituir una fuente, centralizada, de informaci\'on sobre toda aquella persona que pretenda hacer uso de las facilidades disponibles en la red universitaria. Dicha centralizaci\'on representa un elemento necesario para garantizar el control apropiado del uso de dichas facilidades, ya que la fragmentaci\'on de los datos, inherente a la estructura actual de la Universidad y a sus actividades, d\'igase la composici\'on por facultades, departamentos de investigaci\'on, etc, as\'i como la continua realizaci\'on de eventos para fomentar la investigaci\'on y el intercambio de experiencias, aumenta considerablemente la complejidad a la hora administrar la informaci\'on almacenada y la restricci\'on o concesi\'on de acceso a los servicios brindados. 

Sin embargo, esta informaci\'on no se genera de forma centralizada, sino que se encuentra esparcida entre distintas fuentes como son el SIGENU, .... \todo[color=yellow]{Cu\'ales son los nombres oficiales de las fuentes de datos?, Qu\'e significa SIGENU? si es que son siglas} 
Por esta raz\'on, esta informaci\'on centralizada, es actualizada a trav\'es de diversos procesos con una frecuencia no tan regular como debiera ser, lo cu\'al se debe principalmente a que parte de estos procesos mencionados anteriormente, ni siquiera son realizados autom\'aticamente, sino que existe una persona(o varias) encargada transportar, generalmente en medios f\'isicos, la informaci\'on desde donde se genera hasta donde es necesitada (en casa del herrero, cuchillo de palo).

Entre los datos m\'as relevantes, almacenados en este sistema, se encuentra aquellos que permiten la restricci\'on y/o concesi\'on de acceso a los servicios brindados, es decir, las credenciales de los usuarios de la red. Esta informaci\'on permite:

\begin{enumerate}
\item {\bf Autenticar al usuario:} Comprobar que la persona que solicita un servicio es quien dice ser.
\item {\bf Administrar el acceso:} En correspondencia del rol de la persona autenticada, permitir o no el acceso a ciertos servicios
\end{enumerate}
   