\chapter{Directorio \'Unico}

\section{Surgimiento del Directorio \'Unico}

Como resultado de la necesidad de acceso a la informaci\'on de los usuarios de la red de la Universidad de la Habana, tanto de estudiantes, como profesores y externos, varios servicios han sido implementados de manera aislada, durante los \'ultimos años con el fin de suplir esta necesidad. A ra\'iz del proceso de centralizaci\'on de los servicios llevado a cabo hace ya tres años, dichos servicios, relacionados con la creaci\'on, almacenamiento y la actualizaci\'on de la informaci\'on referente a los usuarios, fueron integrados en lo que se conoce hoy como Directorio \'Unico.

Entre las principales funcionalidades que componen el Directorio \'Unico, se encuentra la de constituir una fuente, centralizada, de informaci\'on sobre toda aquella persona que pretenda hacer uso de las facilidades disponibles en la red universitaria. Dicha centralizaci\'on representa un elemento necesario para garantizar el control apropiado del uso de dichas facilidades, ya que la fragmentaci\'on de los datos, inherente a la estructura actual de la Universidad y a sus actividades, d\'igase la composici\'on por facultades, departamentos de investigaci\'on, etc, as\'i como la continua realizaci\'on de eventos para fomentar la investigaci\'on y el intercambio de experiencias, aumenta considerablemente la complejidad a la hora de administrar la informaci\'on almacenada y de garantizar cierta restricci\'on o concesi\'on de acceso a los servicios brindados. 

Sin embargo, esta informaci\'on no se genera de forma centralizada, sino que se encuentra esparcida entre distintas fuentes como son el SIGENU y una base de datos conocida como 'Assets', la cual se encuentra esparcida en varios servidores y sobre la cual no se tiene total acceso debido a que, para su creaci\'on y luego su administraci\'on, se utiliza un software privativo, el cual impide el acceso requerido para implementar las actualizaciones que se hacen necesarias en este momento. \todo[color=yellow]{Cu\'ales son los nombres oficiales de las fuentes de datos?, Qu\'e significa SIGENU? si es que son siglas} 
Por esta raz\'on, para poder actualizar esta informaci\'on centralizada, todos los datos existentes en las fuentes antes mencionadas, son procesados a trav\'es de un sistema que se compone de alrededor de 3 o 4 capas de procesamiento. Entre las principales razones que justifican la existencia de estas capas, se encuentra la incapacidad de agregarle directamente a las fuentes de datos, un campo que indique en que momento una instancia de usuario ha sido modificada.

Entre los datos m\'as relevantes, almacenados en este sistema, se encuentra aquellos que permiten la restricci\'on y/o concesi\'on de acceso a los servicios brindados, es decir, las credenciales de los usuarios de la red. Esta informaci\'on permite:

\begin{enumerate}
\item {\bf Autenticar al usuario:} Comprobar que la persona que solicita un servicio es quien dice ser.
\item {\bf Administrar el acceso:} En correspondencia del rol de la persona autenticada, permitir o no el acceso a ciertos servicios
\end{enumerate}

Del resto de los datos que actualmente se mantienen en el sistema, una parte se reserva para brindar informaci\'on acerca de la actividad realizada por el usuario durante su estancia en la red, es decir, los momentos en que se autentic\'o en la red, en que sistema lo hizo, la cantidad de cuota de internet consumida, etc. La otra parte, estando esta \'ultima m\'as asociada a los trabajadores y externos, contiene informaci\'on m\'as personal de los usuarios. Entre la misma sobresalen: 

\begin{enumerate}
	\item Año que cursa (en caso de ser estudiante)
	\item Direcci\'on Particular
	\item Departamento al que pertenecen
	\item Datos sobre la nomina
	\item Puesto que ocupa
	\item Cargos importante (si es que los pos\'ee)
\end{enumerate}

Esta informaci\'on es requerida principalmente por los departamentos responsables de la administraci\'on de los recursos humanos de la Universidad de la Habana.

\section{Problemas del sistema actual}

El sistema, tal y como existe en este momento, presenta varios problemas, a lo cuales esta tesis pretende brindar una posible soluci\'on. Los problemas son presentados a continuaci\'on:

\subsection{Mantenimiento y extensi\'on del Sistema}

Debido a la naturaleza del surgimiento del Directorio \'Unico, es decir, el acoplamiento de varios servicios aislados en uno, cada componente es demasiado dependiente de la forma en que las dem\'as brindan sus correspondientes funcionalidades. Esto se debe a que la interacci\'on entre las mismas ha sido configurada mediante un enfoque est\'atico \todo[color=yellow]{Mi intenci\'on aqu\'i es decir que estan cableadas pero realmente no se como decirlo}, y que el propio sistema no dispone de herramientas que permitan su modificaci\'on de una manera c\'omoda para los encargados de su mantenimiento. Dicho enfoque dificulta enormemente las tareas de actualizaci\'on del sistema, las cuales son necesarias para poder adecuar el mismo a las nuevas condiciones y necesidades que van surgiendo en la red a trav\'es de los años.

\subsection{Desuso de datos almacenados}

Desde el surgimiento del Directorio \'Unico, se han ido incorporando nuevos campos al diseño de las bases de datos que representan las fuentes de informaci\'on del sistema, con el fin de suplir las necesidades que ocupan a la Universidad en cada nuevo per\'iodo escolar. 

Actualmente muchos de esos campos han dejados de ser \'utiles para la Universidad, pero a pesar de esto se mantiene en el sistema debido a que, como consecuencia de la rigidez del mismo, cualquier cambio, sobre todo aquellos cuya repercusi\'on y alcance no se conocen, podr\'ian significar la ca\'ida del sistema por tiempo indefinido, de ah\'i que se siga la filosof\'ia de que "... lo que funciona no se toca...", provocando una sobre carga innecesaria para el sistema, que aunque pueda ser pequeña, no deja de ser significante, reflej\'andose sobre todo en el espacio ocupado por la informaci\'on en disco.

\subsection{Incumplimiento de pol\'iticas de baja de usuarios}

Este es otro problema en el cual se incurre con bastante frecuencia en la Universidad. Debido a la volatilidad de algunos contratos concertados con personal ajeno a nuestro centro de altos estudios, sucede frecuentemente que al dar de baja a estos usuarios, sus cuentas son eliminadas con efecto casi inmediato, lo cual va en contra de los protocolos usualmente implementados en estos casos, los cuales son necesarios para prevenir la p\'erdida de acceso a servicios cr\'iticos, muchas veces desarrollados, administrados y mantenidos por agentes externos a la Universidad.		


\subsection{Carencia de r\'eplicas sincronizadas }

Actualmente el Directorio \'Unico no cuenta con r\'eplicas que puedan garantizar el continuo funcionamiento del sistema en caso de que el servicio principal se detenga por causa de alg\'un problema en el mismo. Esto causa la imposibilidad de acceder a gran parte de los servicios pertenecientes a la red universitaria, afectando la productividad de la misma.
Este problema no se debe solo a la falta de r\'eplicas del servicio actual, si no tambi\'en a la falta de alternativas para este servicio, ya que actualmente no existe ninguna funcional, a pesar de que el Directorio \'Unico lleva fallando ya un buen tiempo.

\subsection{Carencia de protocolos comunes para servicios externos}

Resulta natural que, con el paso del tiempo, se haga necesaria la disponibilidad de nuevos servicios en la red de la Universidad. Generalmente estos servicios, presentan como requerimiento com\'un, aquel asociado a la autenticaci\'on de usuarios, as\'i como a la administraci\'on de roles para los mismos. Por esta raz\'on, esta tarea se suele asignar a un sistema externo y centralizado con el objetivo de evitar que se repita este proceso en el desarrollo de cada servicio. Generalmente es implementado utilizando uno de los protocolos desarrollado por las entidades encargadas de estandarizar ciertos procesos en el amplio campo de la ciencia de la computaci\'on. Teniendo esto en cuenta, es que se pretende implementar una API Rest que permita modificar la l\'ogica detr\'as de la informaci\'on brindada, sin que esto implique modificar todos los servicios que consuman informaci\'on de nuestro sistema.





