\begin{introduction}
Desde la propuesta de Leonard Kleinrock en el año 1961 en un artículo titulado \todo[color=red]{Lleva referencia} \verb@"Information Flow in Large Communication Nets"@ (traducido ha
\verb@"Flujo de Información en Grandes Redes de Comunicación"@), y el uso del término
\todo[color=red]{Lleva referencia}\verb@"paquete"@ en 1965 por Donald Davies para describir datos enviados entre computadoras
en una red, \todo[color=yellow]{Fue la invención del término lo que provocó el impulso de ARPANET? Además, esto realmente viene al caso?}que impulsarían el desarrollo de ARPANET entre 1966 y 1969, \todo[color=orange]{No me parece adecuado el uso de la palabra planeta aca. Realmente fue solo la humanidad (parte de ella, ni siquiera me queda claro que porciento)} el planeta 
entró en una nueva etapa que revolucionó el desarrollo en todos los campos 
tecnológicos: la \verb@"Era Digital"@ o \verb@"Era de la Información"@. Esta era gira
en torno a las nuevas tecnologías e Internet y está llevando a cabo \todo[color=yellow]{Que cambios? Como calificas cuales de ellos son profundos?}cambios profundos y
transformaciones en una sociedad donde \todo[color=cyan]{Este es el resultado natural de la automatización.}la automatización de los procesos mejora
considerablemente la calidad, rapidez y robustez de los mismos, y donde la conectividad
mediante las redes de dispositivos (no puede hablarse solo de computadoras)  facilitan la comunicación y la organización tanto de personas como de empresas y organismos. 

\todo[inline, color=purple]{Las oraciones en el párrafo anterior son largas y la idea detrás de ellas es dificil de seguir. Todo lo que este en un formato similar al del párrafo anterior debe ser separado en oraciones cortas y conexas.}

Una de las ventajas que trae consigo el establecimiento de redes de dispositivos es la posibilidad de descentralizar el almacenamiento de información \todo[color=olive]{Recomiendo cambiar por: 'sin comprometer el acceso'} manteniendo un adecuado acceso a la misma. Actualmente la red de la Universidad de La Habana cuenta con un \todo[color=yellow]{gran número?}gran número de servicios consumidores o fuentes de diferentes tipos de información, pero carece de un servicio que \verb@"administre"@ el flujo de comunicación con los mismos. De estos servicios se benefician no solo las facultades pertenecientes a la Universidad, también se benefician otras instituciones asociadas a la misma como \todo[color=yellow]{Ahora mismo no lo hacen, parte del objetivo de esta Tesis es sentar las bases para que lo hagan.} el Instituto Superior de Diseño (ISDI) y el Instituto Superior de Ciencias y Tecnologías Aplicadas (InSTEC). La gran mayoría  de estos servicios requiere que el la verificación previa del usuario que solicita acceder o \todo[color=yellow]{Agregar informacion no requiere de esta autenticación} agregar información a los mismos, por lo que con el objetivo de centralizar el proceso de verificación se han ido acumulando sobre un dominio web (directorio.uh.cu) varias actualizaciones  aisladas que conforman lo que se conoce como "Directorio Único de la Universidad de La Habana" para poder integrar y administrar servicios que funcionan sobre diferentes tecnologías. \todo[color=yellow]{Reescribir por favor.} Esto  trae consigo dificultad a la hora de integrar nuevos servicios al directorio, así como inestabilidad y falta de robustez en el servicio de verificación y solicitud de información almacenada.



\todo[inline, color=purple]{El párrafo del la explicación del problema debería ser re escrito para su mejor comprensión.}

Dicho esto, el objetivo de este proyecto es proveer a la Universidad de La Habana de un servicio que se encargue de administrar \todo[color=yellow]{Si vas a usa esto como un objetivo general, 'estos procesos' debe ser claramente definido. Cuales son los procesos?}estos procesos de manera confiable y eficiente,
que además sea fácil de integrar y fácil de modificar (implementar mejoras).

Para lograr esto surgen varias preguntas a responder como por ejemplo: ¿qué método 
se debería utilizar para almacenar la información?, ¿qué método se debería utilizar 
para acceder a la misma?, ¿qué método usar para mantener la informacón actualizada?... \todo[color=yellow]{Este es el punto de conducir la investigación, es mejor desarrollar una separación lógica de estas preguntas mas que presentarlas planamente al lector.}

\todo[color=yellow]{Estas son las respuestas que deberiamos ir intercalando con las preguntas del parrafo anterior.}Como respuesta a la primera pregunta, después de analizar distintos métodos para gestionar información, debido a las características del problema que enfrentamos llegamos a la conclusión que lo \todo[color=orange]{Esta afirmacion necesita una referencia que la apolle. No tiene que ser a un artículo, puede ser a la sección del documento donde se explica la razón y dicha sección tendria referencias a cualquier otro documento.}mejor es almacenarla en una estructura de directorios. La principal ventaja de esto sobre las bases de datos relacionales (y no relacionales) es la velocidad para realizar consultas sobre grandes volúmenes de datos (perdiendo en velocidad de modificación de los mismos, tanto en actualizaciones, como en inserciones y eliminaciones).

\newpage

\todo[color=yellow]{Evitemos usar \textbf{Debido a esto}  y \textbf{Decido esto}, se puede reformular como \textbf{Habiendo analizado las ventajas de usar una estructura de directorio se decidió ...}}Decidido esto y después de analizar diversas maneras de manejar una estructura de directorios optamos por el LDAP (Lightweight Directory Access Protocol o Protocolo Ligero de Acceso a Directorios) como protocolo de acceso y OpenLDAP como implementación del mismo.

\todo[color=red]{En la introducción no toca explicar el protocolo LDAP. Tiene que haber una sección del capitulo 2 para explicar el concepto de LDAP y como funciona.}Este protocolo  brinda un esquema similar al de una guía telefónica, implementando un enfoque jerárquico para el almacenamiento de la información. \todo[color=orange]{Esto esta aqui para plantear algun objetivo?}Además existe una interfaz de autenticación para la mayoría de los servicios que administran usuarios y permisos, que utiliza como fuente de información a servidores que implementen este protocolo. 
Partiendo de la problemática existente la autenticación de los servicios que se brindan en el recinto universitario, se ha formulado la siguiente hipótesis de investigación: \todo[color=red]{Generar la hiposteis a partir del objetivo que puede ser: \textbf{la implementacion de un sistema LDAP para la administración de la información dentro de la UH} y hacer notar que de las deficiencias citadas (que debieron haber sido mencionadas anteriormente) saldrán las mejoras inmediatas}mediante el sistema de autenticación basado en protocolo LDAP se solucionará la administración eficiente de la información y los servicios en la Universidad de La Habana.

Con esta tesis queremos sustituir el actual sistema de directorio único por un sistema de autenticación basado en el protocolo LDAP.

\todo[color=green]{Esta último parrafo esta bien.}Para esto es necesario investigar sobre el estado del arte de los sistemas que implementan este protocolo, así como la disponibilidad que brindan. Además es necesario automatizar el proceso de recopilar la información provista por estas fuentes, para lo cual se diseñará una esquema de base de datos, guiado por el protocolo LDAP, que se adapte a la estructura de los datos almacenados actualmente. Por último, se plantea implementar una \textit{API} acorde con el protocolo OpenId para facilitar la autenticación para futuros servicios.

\listoftodos[Notes]

\end{introduction} 